\documentclass[]{article}
\usepackage{amsmath}
\usepackage{amssymb}
\usepackage{physics}

%opening
\title{Appunti di Meccanica Quantistica}
\author{Andrea Martinelli}

\begin{document}

\maketitle

\begin{abstract}
\end{abstract}



\section{Concetti matematici}
Partiamo da alcuni concetti matematici di base. Sia $A \in M_{m,n}[ \mathbb{C}]$ una matrice a coefficienti complessi, ovvero $a_{m,n} \in \mathbb{C} \forall m,n$. Allora definiamo le seguenti:
\subsection{Matrice aggiunta}
Si definisce $Acroce$ l'aggiunta di $A$ la matrice che per elementi il trasposto e coniugato di A, ovvero:
\begin{equation}
	\big(A\big)^+_{m,n} = \overline{\big(A\big)_{n,m}}
\end{equation}
\subsection{Matrice hermitiana}
\subsection{Prodotto scalare}
	\paragraph{Relazioni varie}
		
	\paragraph{Spettro reale}
	\paragraph{Base ortogonale}

\subsection{Matrice unitaria}
	\paragraph{Teoremi}
	
\subsection{Commutatore}
	\paragraph{Proprietà}
	\paragraph{Teoremi}
	
	
\section{Calcolare la probabilità}
	
\section{Autostati come base}
	\paragraph{Postulato di Von Neumann}
	
\section{Operatori associati a grandezze fisiche}

\section{Valori di aspettazione}
	\paragraph{Varianza}
	
\section{Teorema di indeterminazione: principio di indeterminazione di Heisemberg}
	\paragraph{Proprietà operatori q, p}
	
	
\section{Operatore Hamiltoniano nel commutatore}








\section{Oscillatore armonico quantistico}
Supponiamo di essere nel caso di un oscillatore armonico, con $\omega = \sqrt{\frac{K}{m}}$ definito dai vettori (monodimensionali) $(q, p)$. La sua hamiltoniana sarà:

\begin{equation}
	H(q, p) = K(q, p) + V(q, p) = \frac{p^2}{2m} + \frac{1}{2}m\omega^2q^2
\end{equation}

Sapendo che l'energia $E \geq 0$, ci chiediamo quale sia lo spettro degli autovalori dell'operatore hamiltoniano $\hat{H}$, ovvero l'insieme dei valori $E \in \mathbb{R}$ tali che:

\begin{equation}
	\hat{H} \ket{E} = E \ket{E}
\end{equation}

Supponiamo ancora che lo stato $\ket{E}$ sia descritto da una base ortonormale.
Quindi $\bra{E}\ket{E} = 1$. Osserviamo che:

\begin{equation}
	\bra{E}\hat{H}(q, p)\ket{E} = \bra{E}\big(\frac{p^2}{2m} + \frac{1}{2}m\omega^2q^2\big)\ket{E} = \bra{E}\frac{p^2}{2m}\ket{E} - \bra{E}\frac{1}{2}m\omega^2q^2\ket{E}
\end{equation}

Osserviamo che:

\begin{equation}
	\bra{E}\hat{H}(q, p)\ket{E} = \bra{E} \big(\hat{H} \ket{E} \big) = E \bra{E}\ket{E} = E \geq 0
\end{equation}

Ovvero l'energia $E$ altro non è che il valore medio dell'operatore hamiltoniano.

\newpage
Eguagliando la (4) e la (5) segue che:

\begin{equation}
	\bra{E}\frac{p^2}{2m}\ket{E} - \bra{E}\frac{1}{2}m\omega^2q^2\ket{E} \geq 0
\end{equation}

Per calcolare la seguente espressione dobbiamo calcolare $q^2, p^2$, in particolare:

\begin{cases}
	
\end{cases}


\end{document}
