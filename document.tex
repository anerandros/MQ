\documentclass[]{article}
\usepackage{amssymb}

%opening
\title{Appunti di Meccanica Quantistica}
\author{Andrea Martinelli}

\begin{document}

\maketitle

\begin{abstract}
\end{abstract}



\section{Concetti matematici}
Partiamo da alcuni concetti matematici di base. Sia $A \in M_{m,n}[ \mathbb{C}]$ una matrice a coefficienti complessi, ovvero $a_{m,n} \in \mathbb{C} \forall m,n$. Allora definiamo le seguenti:
\paragraph{Matrice aggiunta}
Si definisce $Acroce$ l'aggiunta di $A$ la matrice che per elementi il trasposto e coniugato di A, ovvero:
\begin{equation}
	\big(A\big)^+_{m,n} = \overline{\big(A\big)_{n,m}}
\end{equation}
\paragraph{Matrice hermitiana}
\paragraph{Prodotto scalare}






\end{document}
